\documentclass[usenames,dvipsnames]{beamer}
\usepackage{amsmath}
\usepackage{mathtools}
\usepackage{bussproofs}
\usetheme{metropolis}
\usefonttheme[onlymath]{serif}
\title{Harmony and Pitch Groups}
\date{September 18th, 2020}
\author{Ali Taqi}
\institute{Reed Student Colloquium}

\newcommand{\N}{\mathbb{N}}
\newcommand{\Q}{\mathbb{Q}}
\newcommand{\Z}{\mathbb{Z}}
\newcommand{\Prms}{\mathbb{P}}
\newcommand{\octave}{\sim_{8\text{ve}}}
\newcommand{\harm}{\text{Harm}}


\newcommand{\Hz}{\text{Hz}}

\begin{document}

  \maketitle
  \section{Origins: The Integers (500BC)}

  \begin{frame}{Integers}
  "God created the integers; all else is the work of man." -Leopold Kronecker \newline
  The integers are a set of numbers, defined as follows: \newline
  $$\mathbb{Z} = \{...,-2,-1,0,1,2,...\}$$
  \newline
  However, the set of numbers we are more interested in is the natural numbers. \newline
  $$\mathbb{N} = \{1,2,3,...\}$$
  \end{frame}

  \begin{frame}{Harmonics}
One of the primary elements of music is pitch. A pitch, measured in $\Hz (s^{-1})$, is a resonance at a particular frequency, say $f$. \newline
  Example: $\lambda = \,$A4 $\Rightarrow f_\lambda = 440\Hz$ \newline
  \end{frame}
  
  \begin{frame}{Harmonics}
  In nature, there exists something called the Harmonic series. From physics, we know that a resonant body doesn't really ever resonate at a pure frequency. Instead, the frequency we perceive it as is called the fundamental frequency. 
  The harmonic series is defined as the product of some fundamental frequency $f$ and the series of natural numbers $\N$. Define the $\textbf{harmonic series}$ of $f$ to be the set:\newline 
  $$\text{Harm}[f] = \{nf : n \in \mathbb{N}\}$$
  \end{frame}
  
  \begin{frame}{Harmonics: Example}
      Let us dissect this notation. Let us take the same note as before, an octave lower. That is, let $\lambda = \space$ A3. So, $f_\lambda = 220\Hz$. Now, recall that the natural numbers are defined as the set $\mathbb{N} = \{1,2,3,...\}$. From our definition of the set of harmonics, we obtain:
  \begin{align*}
  \text{Harm}[f_\lambda] &= \{1\cdot220\Hz, 2\cdot220\Hz, 3\cdot220\Hz, ...\} \\
  &= \{220\Hz, 440\Hz, 660\Hz, ...\}
  \end{align*}
  \end{frame}
  
  \begin{frame}{Consonance: The Major Triad}

Now, we have the tools to start defining some basic musical objects. Take the subset $A  = \{1,3,5\} \subset \N$. Then the harmonic subset with respect to $A$ can be defined as: 
  \begin{align*}
  \text{Harm}_A [f] &= \{af : a \in A\} \\
			    &=\{f,3f,5f\}
   \end{align*}

The subset A is precisely the holy grail of consonance, at least in the Western canon: the major triad. The connection between our modern day musical intervals and harmonics will be discussed later. There is also a caveat, which is that there exists no practical tuning system perfectly captures tonic consonance using A for all elements non-trivial scale.

  \end{frame}
  
    \begin{frame}{Harmonics: Primes}
The prime numbers is a subset of $\mathbb{N}$, and we denote it the set $\mathbb{P} \subset \mathbb{N}$. Since $\mathbb{P}$ is a proper subset of a naturals, we may derive a unique set of harmonics, called the $\textbf{characteristic harmonics}$ of a fundamental $f$. Namely,
$$   \text{Harm}_\Prms [f] = \{pf : p \in \mathbb{P}\} $$
To understand why this is significant set, we will motivate the next definition, octave equivalency.
  \end{frame}
  
  \begin{frame}{Harmonics: Octave Equivalency}
  We may define an equivalence relation $\octave$ on the space of fundamental frequencies, called $\textbf{octave equivalency}$, as follows:
  $$ f  \octave \hat{f} \iff \exists K \in \Z : \hat{f} = 2^{K} f $$
  The beauty of this relation is that it is cross-cultural, agreed upon in the musical traditions of almost every culture if not all. The quick reasoning for this is because doubling some frequency $f$ is the same action as taking the first non-fundamental element from its harmonic series, $\harm [f]: 2f$.
  \end{frame}
  
  \begin{frame}{Intervals and the Rational Numbers}
  Now that we have motivated octave equivalency, we may finally expand our musical horizons. Notice, that in the definition of octave equivalency, it suffices to find any $K \in \Z$ such that one frequency is $2^K$ multiples of the other for them to be equivalent. \newline Up until now, our definition of harmonics only utilized the multiplication of frequencies by positive whole numbers $(n \in \N)$. However, as we can see, octave equivalency already requires us to use multiplicative inverses, or in other words, fractions.
  \end{frame}
  
    
  \begin{frame}{Intervals and the Rational Numbers}
  Now, we may define an extremely powerful and versatile musical object: the $\textbf{justly-intonatated interval}$, from a tonic frequency $f$ is element the set: 
  
  $$ \mathbb{I}[f] = \left( \frac{p}{q} \cdot f : \frac{p}{q} \in \Q^+ \backslash \{0,1\} \right) $$
  
  The set of intervals may be further partitioned as follows:
  An interval is $\textit{descending}$ if its interval scalar $pq^{-1} \in (0,1)$, and $\textit{ascending}$ if $pq^{-1} \in (1,\infty)$.
  
  \end{frame}
  
  \begin{frame}{Euler's Consonance Formula}
  Not all intervals are created equal. 
  \end{frame}
  
  \section{Beauty of Standardization: 12-TET (1700AD)}
  
  \begin{frame}{New horizons: N-TET}
  To solve the problems of just intonation, mathematicians utilized the freshly discovered idea of a logarithm to define a new tuning system. Define a N-TET ($N \in \N$) tuning system about a fundamental $f$ to be the set of frequencies: 
  
  $$ [\text{N-TET}]_f = \{ 2^{\frac{i}{N}} f :  i \in \N \, | \, 0 \leq i \leq N-1 \}$$
  
  It is worthwhile to note that this set is a cyclic group isomorphic to $\Z/N\Z$! This is even more powerful when we drop the restriction on the domain of $i$. That is, if we allow $i \in \Z$, and attach an octave equivalence $\octave$, we are able to access much more frequencies!
  
  Fun fact: Our current system is a N-TET system, with $N = 12$, and it is tuned about $f = 440\Hz$, the note \text{A4}.
  \end{frame}
  
\end{document}